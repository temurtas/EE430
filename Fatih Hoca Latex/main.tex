\documentclass[10pt,a4paper,report]{report}       %a4paper, report format, 15pt letters
\usepackage{amsmath, epsfig, indentfirst, graphics, enumerate} 
\usepackage{graphicx, amsfonts}                              %graphics related package
%\usepackage{color}                                 %you can add colors with this package
\usepackage[colorlinks]{hyperref}
\usepackage{etoc,minitoc, adjustbox}
\usepackage{mathrsfs, relsize}

\setlength{\textwidth}{7in} \setlength{\textheight}{10in}
\setlength{\topmargin}{-.5in} \setlength{\oddsidemargin}{-0.2in}

\parindent 0in
%\parskip 0.1in2

% Definitions for some environments
\newtheorem{define}{Definition}
\newtheorem{theorem}{Theorem}
\newtheorem{lemma}{Lemma}
\newtheorem{proof}{Proof}
\newtheorem{corollary}{Corollary}

\newenvironment{ex}{\textbf{Ex: }} {\vspace{2cm}}

\usepackage{setspace}
%\singlespacing
\onehalfspacing

% definitions
\newcommand{\om}{\omega}


% ------------------------------------------------------------------------
% Document Starts here:
% ------------------------------------------------------------------------
\begin{document}

\begin{titlepage}
    \begin{center}
        \vspace*{1cm}
        \vspace{3.5cm}
        \textbf{{\Huge Lecture Notes}}
        
        \vspace{1.5cm}
        \textbf{{\Huge EE430 Digital Signal Processing}}
        
        \vspace{0.5cm}
        \vspace{5.5cm}
        \vspace{2.5cm}
        \textbf{{\Large Department of Electrical and Electronics Engineering}}
        
        \vspace{0.5cm}
        \textbf{{\Large Middle East Technical University (METU)}}
    \end{center}
\end{titlepage}

\dominitoc% Initialization
\tableofcontents % Global toc
\large % You can increase or reduce sizes of letters by using such comments.


% ------ New Chapter starts here -----------------------------------------
% ------------------------------------------------------------------------
\chapter{Discrete-time Signals and Systems}
\minitoc% Creating an actual minitoc
\vspace {2cm}

This chapter covers the fundamental concepts of discrete-time (DT) signals and systems. In particular, we cover Sections 2.1 to 2.9 from our textbook.\\


Reading assignment for this chapter :
\begin{itemize}
  \item Sections 2.1 to 2.9 from our textbook.
\end{itemize}

\newpage
%----------------------------------------------------------------------------------------------
\section{Discrete-time (DT) signals}

A DT signal is simply a sequence of numbers indexed by integer $n$.\\
\vspace{3cm}

Our notation to show a DT signal is : \\

A DT signal can be obtained from
\begin{itemize}
  \item an inherently discrete event (e.g. number of students attending lecture $n$)
  \item sampling of a continuous-time (CT) signal :
\end{itemize}
\vspace{3cm}


\subsection{Basic sequences and sequence operations}

\subsubsection{Unit sample sequence}
\vspace{2cm}

Any sequence $x[n]$ can be written in terms of delayed and scaled $\delta[n]$.\\
\vspace{3cm}

For an arbitrary $x[n]$, we have :\\
\vspace{2cm}

\subsubsection{Unit step sequence}
\vspace{2cm}

Relations between unit step and sample sequences:\\
\vspace{3cm}

\subsubsection{Exponential sequences}
General form :\\

If $A$ and $\alpha$ are real, $x[n]$ is real.
\begin{itemize}
  \item $A>0$, $0<\alpha<1$ : $x[n]$ decreases in time
  \item $A>0$, $-1<\alpha<0$ : $x[n]$ increases in time with alternating sign
  \item $|\alpha|>1$ : $x[n]$ grows in magnitude as $n$ increases 
\end{itemize}
\vspace{3cm}

If $A$ and $\alpha$ are complex :
\vspace{8cm}

\subsubsection{Complex exponentials}
\begin{center}
$x[n]=\qquad \qquad \qquad \qquad$\\
\end{center}

Properties :
\begin{enumerate}
  \item Complex exponentials $Ae^{j(\om_0+2\pi r)n}$ with frequencies $(\om_0 + 2\pi r), ~r\in \mathbb{Z}$ (e.g. $\om_0$, $\om_0+2\pi$, $\om_0+4\pi$, ...) are equivalent to each other:\vspace{2cm}
  
  \item Based on above property, when discussing complex exponentials $Ae^{j\om_0n}$ (or sinusoids $\cos(\om_0n+\phi)$), we only need to consider and interval of length $2\pi$ for frequency $\om_0$:\vspace{2cm}
  
  \item Complex exponentials $Ae^{j\om_0n}$ (or sinusoids $\cos(\om_0n+\phi)$) are \underline{periodic only if } $\frac{2\pi}{\om_0}$ is a ratio of integers, i.e. \\
  \vspace{0.5cm} 
  Remember periodicity requirement for any sequence $x[n]$:\\
  \vspace{4cm}
  
  \ex Are following sequences periodic ? If so, find the periods.\\
  $x_1[n]=\cos(n) \qquad \qquad x_2[n]=\cos(\frac{2\pi}{8}n) \qquad \qquad x_3[n]=\cos(\frac{3\pi}{8}n+\phi)$
  \vspace{6cm}
  
  \item (Prop.1 + Prop. 3) There are only $N$ distinguishable frequencies for which the complex exponentials $Ae^{j\om_0n}$ (or sinusoids $\cos(\om_0n+\phi)$) are periodic with $N$: \vspace{4cm}
  
  \item For complex exponentials $Ae^{j\om_0n}$ (or sinusoids $\cos(\om_0n+\phi)$), 
  \begin{itemize}
    \item low frequencies are in the vicinity of $\om_0=$ 
    \item high frequencies are in the vicinity of $\om_0=$
  \end{itemize}
  \vspace{3cm}
  Rate of oscillation of complex exp. (or sinusoid) determines whether frequency is high or low:
  \begin{figure}[h]
    %%\includegraphics[clip,trim=1.4cm  1.8cm  1.4cm  1.2cm, width=1.00\textwidth]{./chp1/figs/Slayt5.JPG}
  \end{figure}
  
\end{enumerate}

\underline{Note :} For CT complex exponential $x(t)=Ae^{j\phi_0t}$, \underline{none} of the above 5 properties hold :
\begin{enumerate}
  \item 
  \item
  \item
  \item
  \item
\end{enumerate}
\vspace{1cm}

\subsubsection{Transformation of independent variable $n$}
Time shift:
\vspace{4.5cm}

Time reversal:
\vspace{4.5cm}

Note : First time shift, then time reversal $\neq$ first time reversal, then time shift\\
\vspace{1cm}


\section{Discrete-time (DT) systems}

\subsubsection{Notation:} \vspace{3cm}

\subsection{Memoryless systems:} 
Output $y[n]$ does not depend on past or future values of input $x[n]$. \\
\ex
\vspace{3cm}

\subsection{Linear systems:} 
The systems satisfies the following relation for any $a, b, x_1[n], x_2[n]$: 
\vspace{4cm}\\
\vspace{3cm}

In a linear system, if input \\ 
\vspace{2cm}

\ex Are these systems linear ?\\
\vspace{9cm}

\subsection{Time-invariant systems:} 
Any time shift at the input causes a time shift at the output by the same amount.\\
\vspace{2cm}

\ex Are these systems time-invariant ? (Accumulator)
\vspace{4cm}

\ex Are these systems time-invariant ? (Compressor)
\vspace{3cm}

\subsection{Causality:}
Current output sample $y[n]$ depends only on current and past input samples $x[n], x[n-1],x[n-2],...$\\
\ex
\vspace{3cm}

\subsection{Stability:}
A system is stable if and only if (iff) every bounded input (i.e. $\qquad \qquad \qquad\qquad\qquad$) produces a bounded output (i.e. $\qquad \qquad \qquad\qquad\qquad$).\\
\ex 
\vspace{6cm}


\section{Linear time-invariant (LTI) systems}
LTI systems have both linearity and time-invariance properties.\\ 
LTI systems are a very important class of systems.\\
The output of a LTI system to an arbitrary input can be calculated by the famous convolution sum:
\vspace{5cm}\\
\vspace{4cm}\\

Hence, for any input $x[n]$ to an LTI system, output \\
\vspace{2cm}

An LTI system is \underline{completely characterized} by its impulse response $h[n]=T\{\delta[n]\}$.
\vspace{4cm}\\
\vspace{4cm}\\

\subsection{Computation of convolution sum:}
\vspace{2cm}
\ex $x[n]=\delta[n+2]+2\delta[n]-\delta[n-3]$ is input to LTI system with impulse response $h[n]=3\delta[n]+2\delta[n-1]+\delta[n-2]$. Find output $y[n]$ using two methods.\\
\vspace{3cm}\\
\underline{Echo method} : Add outputs to each weighted and delayed delta function in the input. (Useful when input has few samples.)\\
\vspace{4cm}\\
\vspace{4cm}\\
\underline{Sliding average method} : Apply definition of convolution sum.\\
\vspace{4cm}\\
\vspace{4cm}

\ex Impulse response of LTI system is $h[n]=u[n]-u[n-N]$ and input $x[n]=a^nu[n], \quad 0<a<1$. Find output $y[n]$.\\
\vspace{3cm}\\
\vspace{4cm}


\section{Properties of convolution and LTI systems}
\subsection{Properties of convolution}
\begin{itemize}
 \item \textbf{Distribution} over addition:\\ \\
 \item \textbf{Commutative} property:\\ \\
 \item \textbf{Associative} property:\\ \\
\end{itemize}
\begin{figure}[ht]
\centering
\begin{minipage}{.45\textwidth}
  \centering
  %\includegraphics[clip,trim=1.4cm  1.8cm  1.4cm  1.2cm, width=0.80\textwidth]{./chp1/figs_textbook_chp2/image_11.jpg}
  %%\includegraphics[width=0.55\textwidth]{./chp1/figs_textbook_chp2/image_11.jpg}
  \caption{}{(Figure 2.11 in textbook) (a) Parallel combination of LTI systems (b) an equivalent system.}
  \label{fig:test1}
\end{minipage}%
\begin{minipage}{.45\textwidth}
  \centering
  %\includegraphics[clip,trim=1.4cm  1.8cm  1.4cm  1.2cm, width=0.80\textwidth]{./chp1/figs_textbook_chp2/image_12.jpg}
  %%\includegraphics[width=0.55\textwidth]{./chp1/figs_textbook_chp2/image_12.jpg}
  \caption{}{(Figure 2.12 in textbook) (a) Cascade combination of LTI systems (b) equivalent cascade system (c) single equivalent system.}
  \label{fig:test2}
\end{minipage}
%\caption{A}
\end{figure}

\subsection{Properties of LTI systems}
\begin{itemize}
 \item \textbf{Impulse response property:} An LTI system is completely characterized/specified/determined by its impulse response $h[n]$.\\ $\Longrightarrow$ \\
 \item \textbf{Memory property:} LTI system is memoriless $\Longleftrightarrow$ \\
 \vspace{2cm}
 \item \textbf{Causality property:} LTI system is causal $\Longleftrightarrow$ \\
 \vspace{2cm}
 \item \textbf{Stability property:} LTI system is stable $\Longleftrightarrow$ \\
 Proof given in two steps.\\
 Step-1 : Sufficiency, i.e. if $\sum_{n=-\infty}^{\infty}|h[n]|<\infty$, then LTI system is stable.\\ 
 \vspace{3cm}\\
 Step-2 : Necessity, i.e. for LTI system to be stable, we must have $\sum_{n=-\infty}^{\infty}|h[n]|<\infty$.\\
 \vspace{5.5cm}\\
 \item \textbf{Invertibility property:} LTI system ($h[n]$) is invertible $\Longleftrightarrow$ There is another LTI system ($g[n]$) such that $h[n]*g[n]=\delta[n]$.\\
 \vspace{1.5cm}
\end{itemize}

\subsection{FIR and IIR systems}
\textbf{FIR} : \textbf{F}inite (-duration) \textbf{I}mpulse \textbf{R}esponse ($h[n]$ has finite number of nonzero samples)\\
\hspace*{1cm} \ex \\ \\
\textbf{IIR} : \textbf{I}nfinite (-duration) \textbf{I}mpulse \textbf{R}esponse ($h[n]$ has infinite number of nonzero samples)\\
\hspace*{1cm} \ex \\


\section{Linear constant-coefficient difference equations (LCCDE)}
An important subclass of LTI systems exist, where the input $x[n]$ and output $y[n]$ satisfy an  LCCDE.\\ \vspace{1.5cm}

\ex Accumulator : \\
\vspace{3cm}

\ex LTI system with impulse response $h[n]=\frac{1}{n^2}u[n-1]$. System is LTI, but cannot be represented with and LCCDE.\\

\subsubsection{Initial rest conditions (IRC)}
An LCCDE alone does \underline{not uniquely} specify a system. (i.e. there may be multiple systems satisfying the same LCCDE)
\begin{itemize}
  \item Auxiliary conditions are required together with LCCDE to uniquely specify the system.
  \item Some auxiliary conditions may result in non-LTI system.
  \item The so-called \textbf{"initial rest" auxiliary conditions} lead to a \underline{unique} \underline{LTI and causal} system.\\
  \textbf{Initial rest conditions (IRC)} :\\
  \vspace{1cm}
\end{itemize}

In this course, we are mostly interested in \underline{finding impulse response $h[n]$ of LTI} systems (and not necessarily the output $y[n]$ for an arbitrary $x[n]$, because we can then use convolution..).\\  
$\Longrightarrow$ \\ 

We can find the desired $h[n]$ in two steps :
\begin{enumerate}
  \item Find \textbf{homogeneous solution} from \\
  \vspace{1cm}
  \item Find \textbf{initial conditions} for the homogeneous solution using \\
  \vspace{1cm}
\end{enumerate}

\ex System satisfies : $y[n]-3y[n-1]-4y[n-2] = x[n]+2x[n]$. Find $h[n]$ under IRC.\\
\vspace{3cm}\\
\vspace{3cm}

\subsubsection{Recursive calculation from LCCDE}
The \textbf{recursive nature of LCCDE} are very powerful/useful and can also be used to calculate the output $y[n]$ recursively, under many auxiliary conditions.\\
\ex (Same LCCDE) $y[n]-3y[n-1]-4y[n-2] = x[n]+2x[n]$\\
\vspace{4cm}

\subsubsection{LCCDE of FIR and IIR systems}
\underline{If $N=0$}, in the LCCDE equation, we have\\
\begin{itemize}
  \item no recursion and thus no initial/auxiliary conditions are required to compute output
  \item actually, the LCCDE is in the form of a convolution where\\ \\
\end{itemize}
\underline{If $N>1$}, in the LCCDE equation, we have\\
\begin{itemize}
  \item recursion is required to compute output
  \item if IRC used, system is LTI and causal and IIR due to recursion
\end{itemize}

\subsubsection{Transform domain approaches}
Transform domain approaches are best/useful for LCCDE describing LTI systems.\\
\ex LCCDE : $y[n]-3y[n-1]-4y[n-2] = x[n]+2x[n]$\\
\vspace{4cm}



\section{Frequency domain representation of DT signals and systems}
Consider an LTI system with impulse response $h[n]$ and input $x[n]$. The output $y[n]$ is \\ \vspace{1.5cm}

If $x[n]=e^{j\om n}$ for $-\infty < n < \infty$ (i.e. complex exponential with frequency $\om$) \\ 
\vspace{1.5cm}\\
\vspace{3cm}\\
$\Longrightarrow$ \textbullet $~$ $e^{j\om n}$ is the \textbf{eigenfunction} for all LTI systems.\\
\hspace*{0.71cm}  \textbullet $~$ The corresponding \textbf{eigenvalue} is $H(e^{j\om})$, also called the \textbf{frequency response} of the system.\\

\ex What is the frequency response of and ideal delay system ?\\
\vspace{2cm}\\
\vspace{2cm}

It will be shown that a broad class of signals can be represented by a sum of complex exponentials\\
\vspace{1.0cm}

Hence, for an LTI system, the output can be easily calculated\\
\vspace{1.5cm}

Note that the frequency response is periodic with $2\pi$: \\ 
\vspace{0.5cm}%$H(e^{j\om})=H(e^{j(\om+2\pi)})$\\

Therefore, frequency response can be defined only over a range of $2\pi$ :\\
\vspace{0.5cm}

Note that the signals $e^{j\om n}$ and $e^{j(\om +2\pi)n}$ are equal and hence the system cannot differentiate between these eigenfunctions.\\

\ex Input to an LTI systems is $x[n]=A\cos(\om_0n+\phi)$. Find output in terms of $H(e^{j\om})$.\\
\vspace{2cm}\\
\vspace{1cm}\\ 
\vspace{2cm}\\ 

An important class of LTI systems, called \textbf{frequency selective filters}, have frequency response $H(e^{j \om})$ that is unity (i.e. 1) over a range of frequencies and  0 for the remaining frequencies.
\begin{figure}[h!]
\centering
\begin{minipage}{.57\textwidth}
  \centering
  %%\includegraphics[width=0.95\textwidth]{./chp1/figs_textbook_chp2/image_17.jpg}
  \caption{}{(Figure 2.17 in textbook) Ideal lowpass filter showing (a) periodicity of frequency response and (b) one period of frequncy response.}
  \label{fig:test1}
\end{minipage}%
\begin{minipage}{.41\textwidth}
  \centering
  %%\includegraphics[width=0.85\textwidth]{./chp1/figs_textbook_chp2/image_18.jpg}
  \caption{}{(Figure 2.18 in textbook) Ideal frequency selective filters (a) Highpass filter (b) Bandstop filter (c) Bandpass filter.}
  \label{fig:test2}
\end{minipage}
%\caption{A}
\end{figure}

\ex Moving average system: $y[n]=\frac{1}{M_1+M_2+1}\sum_{k=-M_1}^{M2}x[n-k]$.
$~$LTI ? If so, find $h[n]$ and $H(e^{j\om})$.\\
\vspace{2cm}\\
\vspace{2cm}\\
\vspace{3cm}\\
\vspace{2cm}\\

\subsubsection{Suddenly applied complex exponential inputs}
This subsection (Sec. 2.6.2 in textbook) is a reading assignment. It discusses LTI system when inputs are of the form $x[n]=e^{j\om_0 n}u[n]$ instead of $x[n]=e^{j\om_0 n}$.\\
\vspace{2cm}


\section{Representation of sequences by Fourier transforms}

\section{Symmetry properties of DT Fourier transforms}

\section{DT Fourier transform theorems}



\newpage
%% ----------------------------------------------------------------------------
% ---------------------WRITE YOUR PART HERE -----------------------------------
(WRITE YOUR PART HERE)\\



%% ----------------------------------------------------------------------------

\end{document}
% ------------------------------------------------------------------------
% Document Starts here:
% ------------------------------------------------------------------------